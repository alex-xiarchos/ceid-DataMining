\chapter{ΕΡΩΤΗΜΑ 1}
    Το σύνολο δεδομένων περιλαμβάνει 22 \texttt{.csv} αρχεία που αντιστοιχούν σε 22 συμμετέχοντες. Σύμφωνα με την περιγραφή του dataset,
    περιλαμβάνεται η στήλη \texttt{timestamp}, με την ημερομηνία και ώρα, οι στήλες \texttt{back}\(_{x,y,z}\) και \texttt{thigh}\(_{x,y,z}\)
    με τις τιμές του κάθε αισθητήρα για κάθε διάσταση, και η στήλη \texttt{label}, η οποία προσδιορίζει τη δραστηριότητα του συμμετέχοντα τη δεδομένη στιγμή.

    Η στήλη \texttt{label} παίρνει τις εξής τιμές:

    \vspace{-10pt}
    \begin{multicols}{2} \centering \tt
        \begin{itemize}[label={}]
            \addtolength\itemsep{-4mm}
            \item 1 - Walking
            \item 2 - Running
            \item 3 - Shuffling
            \item 4 - Stairs (ascending)
            \item 5 - Stairs (descending)
            \item 6 - Standing
            \item 8: lying
            \item 13 - Cycling (sit)
            \item 14 - Cycling (stand)
            \item 130 - Cycling (sit, inactive)
            \item 140 - Cycling (stand, inactive)
        \end{itemize}
    \end{multicols}
    \vspace{-10pt}

    Για την εισαγωγή και την προεπεξεργασία των αρχείων, θα χρησιμοποιήσουμε τη βιβλιοθήκη \texttt{pandas}
    ενώ για την οπτικοποίηση τις βιβλιοθήκες \texttt{matplotlib} και \texttt{seaborn} της Python.

    \section{ΑΝΑΛΥΣΗ ΣΥΝΟΛΟΥ ΔΕΔΟΜΕΝΩΝ}
        Εισάγουμε τα \texttt{.csv} αρχεία μέσω της \texttt{os} βιβλιοθήκης και της \texttt{read\_csv()} συνάρτησης.
        Καταρχάς, χρησιμοποιώντας τη \texttt{head()} μπορούμε να δούμε τις πρώτες εγγραφές του dataset μας.
        Για παράδειγμα για το πρώτο αρχείο του συνόλου δεδομένων \texttt{S006.csv}:

        \begin{table}[ht] \noindent\centering\tt
        \resizebox{\textwidth}{!}{
            \begin{tabular}{llrrrrrrr}
            & timestamp & back\_x & back\_y & back\_z & thigh\_x & thigh\_y & thigh\_z & label \\
            \midrule
            0 & 2019-01-12 00:00:00.000 & -0.760242 & 0.299570 & 0.468570 & -5.092732 & -0.298644 & 0.709439 & 6 \\
            1 & 2019-01-12 00:00:00.010 & -0.530138 & 0.281880 & 0.319987 & 0.900547 & 0.286944 & 0.340309 & 6 \\
            2 & 2019-01-12 00:00:00.020 & -1.170922 & 0.186353 & -0.167010 & -0.035442 & -0.078423 & -0.515212 & 6 \\
            3 & 2019-01-12 00:00:00.030 & -0.648772 & 0.016579 & -0.054284 & -1.554248 & -0.950978 & -0.221140 & 6 \\
            4 & 2019-01-12 00:00:00.040 & -0.355071 & -0.051831 & -0.113419 & -0.547471 & 0.140903 & -0.653782 & 6 \\
            \end{tabular}}
        \end{table}

        Μέσω της \texttt{info()} εξάγουμε το συμπέρασμα πώς για κάθε χρονική στιγμή δίνονται οι τιμές των αισθητήρων, αποθηκευμένες ως \texttt{float24},
        στις τρεις διαστάσεις (x, y, z) για τις περιοχές της πλάτης και του μηρού, καθώς και ένα \texttt{int64} για το \texttt{label}.
        Η ίδια μορφολογία παρατηρείται σε όλα τα \texttt{.csv} του συνόλου δεδομένων, με κάποιες διαφοροποιήσεις που θα αναλυθούν στη συνέχεια.

        Παρόλο που στην περιγραφή αναφέρεται πως δεν υπάρχουν missing values, για να ελέγξουμε την ακεραιότητα του dataset, μέσω της συνάρτησης
        \texttt{concat()} ενώνουμε όλα τα 22 αρχεία σε ένα ενιαίο dataframe. Τρέχοντας την \texttt{isnull().sum()}, έχουμε:

        \begin{table}[h] \noindent\centering\tt
        \resizebox{0.23\textwidth}{!}{
            \begin{tabular}{lr}
             & sum \\
            \midrule
            timestamp & 0 \\
            back_x & 0 \\
            back_y & 0 \\
            back_z & 0 \\
            thigh_x & 0 \\
            thigh_y & 0 \\
            thigh_z & 0 \\
            label & 0 \\
            index & 5740689 \\
            Unnamed: 0 & 6323682 \\
            \end{tabular}}
        \end{table}

        Παρατηρούμε πως στις στήλες \texttt{back}\(_{x,y,z}\) και \texttt{thigh}\(_{x,y,z}\), οι οποίες είναι και αυτές που μας ενδιαφέρουν, όντως δεν παρατηρούνται missing values.
        Όμως, έχουν εμφανιστεί \texttt{NaN} τιμές στις στήλες \texttt{"index"} και \texttt{"Unnamed: 0"}, οι οποίες στήλες μάλλον θα εμφανίζονται επιπλέον σε κάποια αρχεία.

        Ελέγχοντας όλα τα αρχεία, η στήλη \texttt{"index"} εμφανίζεται στα αρχεία \texttt{S015.csv} και \texttt{S021.csv} και η στήλη \texttt{"Unnamed: 0"} στο αρχείο \texttt{S023.csv}.
        Έπειτα από έλεγχο, φαίνεται πως πρόκειται για δείκτες αύξουσας αρίθμησης που δεν προσφέρουν κάποια επιπλέον πληροφορία.
        Επομένως, μπορούμε να τις αφαιρέσουμε χρησιμοποιώντας τη συνάρτηση \texttt{drop('όνομα', axis=1)}.

        Χρησιμοποιώντας τη συνάρτηση \texttt{describe()} μπορούμε να υπολογίσουμε βασικές στατιστικές μετρικές για τα δεδομένα μας.
        Η συνάρτηση επιστρέφει ένα dataframe με τις ακόλουθες στήλες:

        \vspace{-3mm}
        \begin{itemize}[label={\tiny \blacksquare}]
            \addtolength\itemsep{-3mm}
            \item \texttt{\textbf{count}}: ο συνολικός αριθμός των μη-μηδενικών τιμών για κάθε στήλη.
            \item \texttt{\textbf{mean}}: ο μέσος όρος των τιμών για κάθε στήλη.
            \item \texttt{\textbf{min}}: η ελάχιστη τιμή για κάθε στήλη.
            \item \texttt{\textbf{25\%}}: η τιμή του 25ου εκατοστημορίου για κάθε στήλη.
            \item \texttt{\textbf{50\%}}: η τιμή του 50ου εκατοστημορίου για κάθε στήλη.
            \item \texttt{\textbf{75\%}}: η τιμή του 75ου εκατοστημορίου για κάθε στήλη.
            \item \texttt{\textbf{max}}: η μέγιστη τιμή για κάθε στήλη.
        \end{itemize}

        Ενώνοντας συγκεντρωτικά τις μετρήσεις όλων των συμμετεχόντων στο \texttt{df\_combined} μέσω της \texttt{concat()},
        αυτά είναι τα \textbf{βασικά συγκεντρωτικά στατιστικά μεγέθη} όπως προκύπτουν από το \texttt{describe()}
        για όλες τις μετρήσεις από τους συμμετέχοντες, έχοντας αφαιρέσει την ετικέτα \texttt{label} μιας και αποτελείται από κατηγορικές τιμές:

        \begin{table}[ht] \noindent\centering\tt
        \resizebox{0.85\textwidth}{!}{
            \begin{tabular}{lrrrrrr}
             & back_x & back_y & back_z & thigh_x & thigh_y & thigh_z \\
            \midrule
            count & 6461328 & 6461328 & 6461328 & 6461328 & 6461328 & 6461328 \\
            mean & -0.884957 & -0.013261 & -0.169378 & -0.594888 & 0.020877 & 0.374916 \\
            std & 0.377592 & 0.231171 & 0.364738 & 0.626347 & 0.388451 & 0.736098 \\
            min & -8.000000 & -4.307617 & -6.574463 & -8.000000 & -7.997314 & -8.000000 \\
            25\% & -1.002393 & -0.083129 & -0.372070 & -0.974211 & -0.100087 & -0.155714 \\
            50\% & -0.974900 & 0.002594 & -0.137451 & -0.421731 & 0.032629 & 0.700439 \\
            75\% & -0.812303 & 0.072510 & 0.046473 & -0.167876 & 0.154951 & 0.948675 \\
            max & 2.291708 & 6.491943 & 4.909483 & 7.999756 & 7.999756 & 8.406235 \\
            \end{tabular}}
        \end{table}

        Ως αρχικές παρατηρήσεις, βλέπουμε πως οι τιμές βρίσκονται στο διάστημα \( [-8, 8]\), ενώ η τυπική τους απόκλιση είναι μικρή,
        το οποίο δείχνει ότι οι μετρήσεις είναι αρκετά συμπυκνωμένες γύρω από τον μέσο όρο που είναι κοντά στο μηδέν.
        Προφανώς ελέγχοντας κάθε συμμετέχοντα ξεχωριστά, μπορεί να διεξαχθούν αντίστοιχα συμπεράσματα.

    \section{ΓΡΑΦΙΚΕΣ ΠΑΡΑΣΤΑΣΕΙΣ}

        Μέσω της \texttt{plotbox()} της \texttt{Matplotlib}, μπορούμε να δημιουργήσουμε
        το διάγραμμα των τιμών της \verb|df_combined| για μια πρώτη οπτικοποίηση των δεδομένων:

        \begin{center}
            \includegraphics[scale=0.6]{df_combined_boxplot}
        \end{center}

        Πέρα από τις προηγούμενες παρατηρήσεις που επιβεβαιώνονται, επιπλέον παρατηρούμε μια συμμετρικότητα κοντά στο μηδέν
        για κάθε διάσταση. Επίσης, χρησιμοποιώντας την \texttt{displot()}, μπορούμε να οπτικοποιήσουμε το πώς κατανέμονται οι τιμές.
        Ενδεικτικά για τον \texttt{S009}:

        \begin{multicols}{2} \centering
            \includegraphics[scale=0.5]{img/distributions/S009_back_distributions}
            \includegraphics[scale=0.5]{img/distributions/S009_thigh_distributions}
        \end{multicols}

        Χρησιμοποιώντας την \texttt{plot()}, μπορούμε να δημιουργήσουμε subplots για τις στήλες \texttt{back}\(_{x,y,z}\)
        και \texttt{thigh}\(_{x,y,z}\). Αυτά, για παράδειγμα, είναι τα subplots για τον συμμετέχοντα \texttt{S015}:

        \begin{multicols}{2} \centering
            \noindent\includegraphics[scale=0.55]{img/distributions/S015_back_distributions}
            \includegraphics[scale=0.55]{img/distributions/S015_thigh_distributions}
        \end{multicols}

        Φαίνεται ότι ο συμμετέχοντας κατά τη διάρκεια της μέτρησης μεταβάλλει τη φυσική του δραστηριότητα (μάλιστα με παρόμοιο τρόπο σε πλάτη και μηρό),
        καθώς υπάρχουν στιγμές που δεν υπάρχουν έντονες διακυμάνσεις των τιμών των μετρήσεων των αισθητήρων και άλλες όπου είναι πιο ενεργός,
        με την τιμή του \texttt{label} να αλλάζει και αυτή. Στις στιγμές που ο συμμετέχοντας δεν κινείται, το \texttt{label}
        φαίνεται να παίρνει την τιμή \texttt{8 - Standing} που επιβεβαιώνει τη στασιμότητά του.

        Από την άλλη, στον συμμετέχοντα \texttt{S029} παρατηρούμε πως η κίνηση της πλάτης δεν ταυτίζεται με την (έντονη) κίνηση των μηρών,
        κάτι που μας οδηγεί στο συμπέρασμα πως ο συμμετέχοντας κάνει ποδήλατο.
        Το γεγονός αυτό επιβεβαιώνεται και από τα spikes του \texttt{label} στις τιμές 100.
        \pagebreak

        \begin{multicols}{2} \centering
            \noindent\includegraphics[scale=0.55]{img/distributions/S029_back_distributions}
            \includegraphics[scale=0.55]{img/distributions/S029_thigh_distributions}
        \end{multicols}

        Τέλος, για τον συμμετέχοντα \texttt{S027}, φαίνεται να έχει μια πολύ έντονη φυσική δραστηριότητα με τη \texttt{label}
        να παραμένει σταθερή με τιμή κοντά στο 2.5, κάτι από το οποίο μπορούμε να συμπεράνουμε πως ο συμμετέχοντας τρέχει:

        \begin{multicols}{2} \centering
            \noindent\includegraphics[scale=0.55]{img/distributions/S027_back_distributions}
            \includegraphics[scale=0.55]{img/distributions/S027_thigh_distributions}
        \end{multicols}

        Τέλος, για τον εντοπισμό συσχετίσεων, μπορούμε να δημιουργήσουμε ένα \texttt{heatmap()} μέσω της \texttt{seaborn}.
        Για παράδειγμα, για τον συμμετέχοντα \texttt{S008} φαίνεται πως υπάρχει μια κάποια συσχέτιση ανάμεσα στις στήλες \verb|back_x| + \verb|back_z|,
        \verb|thigh_x| + \verb|thigh_z| και \verb|back_z| + \verb|thigh_x| ενώ στον \texttt{S015} ανάμεσα στις στήλες \verb|back_x| + \verb|back_z| και \verb|back_y| + \verb|thigh_y|.

        \begin{multicols}{2} \centering
            \noindent\includegraphics[scale=0.55]{img/heatmaps/S008_heatmap}
            \includegraphics[scale=0.55]{img/heatmaps/S015_heatmap}
        \end{multicols}

        Δημιουργώντας heatmap και για το συγκεντρωτικό dataframe \verb|df_combined|, παρατηρούμε μια  αμυδρή συσχέτιση
        ανάμεσα στα \verb|thigh_x| + \verb|back_x| και \verb|thigh_x| + \verb|thigh_z|.

        \vspace{-5pt}
        \begin{center}
            \includegraphics[scale=0.5]{df_combined_heatmap}
        \end{center}

        \begin{graycomment} \centering
            Στο παράρτημα παρατίθενται τα plots και τα heatmaps για όλους τους συμμετέχοντες.
        \end{graycomment}