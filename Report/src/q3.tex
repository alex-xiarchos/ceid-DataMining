\pagebreak
\chapter{ΕΡΩΤΗΜΑ 3}

    Για αρχή, θα χρησιμοποιήσουμε τον KMeans αλγόριθμο της \texttt{sklearn.cluster} για τη συσταδοποίηση.
    Καταρχάς περνάμε τα δεδομένα μέσα από έναν scaler, συγκεκριμένα τον \texttt{StandardScaler()}, ο οποίος τα κανονικοποιεί για να είναι ομοιόμορφα.
    Η διαδικασία αυτή είναι απαραίτητη, μιας και ο αλγόριθμος είναι επιρρεπής στις μικροαλλαγές των δεδομένων, και χρειάζεται η συστηματοποίησή τους.

    Μιας και έχουμε πολυδιάστατα δεδομένα, είναι απαραίτητη η μείωση των διαστάσεων, ώστε να είναι εφικτό να απεικονιστούν στο δισδιάστατο επίπεδο.
    Για αυτό το λόγο χρησιμοποιούμε τον PCA της \texttt{sklearn.decomposition}. Εν τέλει, ο KMeans αλγόριθμος δημιουργεί τις επόμενες 5 συστάδες:

    \begin{center}
        \includegraphics[scale=0.75]{img/Kmeans_clustering}
    \end{center}

    Ένας εναλλακτικός τρόπος για τη συσταδοποίηση είναι να χρησιμοποιήσουμε τον DBSCAN αλγόριθμο πάλι της \texttt{sklearn.cluster}.
    Ο DBSCAN δημιουργεί συστάδες οποιουδήποτε σχήματος, ενώ δεν επηρεάζεται τόσο από τα outliers.
    Δεν προκαθορίζεται ο αριθμός των συστάδων· καθορίζεται ένας \texttt{eps} αριθμός που ρυθμίζει τη μέγιστη απόσταση δύο δειγμάτων που ανήκουν στην ίδια συστάδα.
    Είναι υπολογιστικά πιο αργός, αλλά λέγεται πως επιφέρει καλύτερα αποτελέσματα.

    \begin{center}
        \includegraphics[scale=0.75]{img/DBSCAN_clustering}
    \end{center}

    \begin{graycomment}
        Λόγω του πολύ μεγάλου συνόλου δεδομένων και του μεγάλου χρόνου υπολογισμού του, \\ο αλγόριθμος εκτελέστηκε μόνο σε ένα \texttt{.csv} αρχείο αντί για το \texttt{df\_combined}.
    \end{graycomment}

    \section{ΣΥΜΠΕΡΑΣΜΑΤΑ}

        Αν και ο αλγόριθμος παραμετροποιήθηκε αρκετά και δοκιμάστηκαν διαφορετικές τιμές του \texttt{eps}, πέρα από την πολύ αρχή εκτέλεσή του,
        δε δημιούργησε ξεκάθαρες πλειάδες, παρόλο που εκτελέστηκε σε ένα πολύ μικρότερο σύνολο σε σύγκριση με τον KMeans. Σε αυτό ίσως ευθύνονται οι πολλαπλές διαστάσεις των δεδομένων (curse of dimensionality).
        Αντίθετα ο KMeans δημιούργησε άμεσα ξεκάθαρες πλειάδες σε πολύ σύντομο χρονικό διάστημα σε ένα μεγάλο πλήθος δεδομένων.

        Εν τέλει, λόγω της μορφολογίας του συγκεκριμένου συνόλου δεδομένων, που δεν περιλαμβάνει θόρυβο ή πολλά outliers, ο KMeans ανταποκρίνεται καλύτερα.