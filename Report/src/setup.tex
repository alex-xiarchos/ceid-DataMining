\usepackage{geometry} % Page Dimensions
\geometry{a4paper, total={170mm,260mm}, left=20mm, top=20mm}
\usepackage{fontspec} % Γραμματοσείρες
\usepackage{xcolor} % Χρώματα
\usepackage{colortbl} % Χρωματισμός πινάκων
\usepackage{tabularx} % Πίνακες με μεταβλητό πλάτος στηλών
\usepackage{tabularray} % Πίνακες με πιο ευέλικτη διαμόρφωση
\usepackage{multicol} % Πολλαπλές στήλες (όχι για μαθηματικά)
\usepackage{hyperref} % Σύνδεσμοι στο pdf
\UseTblrLibrary{booktabs}
\usepackage{amstex} % Επιπλέον μαθηματικά εργαλεία
\usepackage{amsmath} % Μαθηματικά εργαλεία
\usepackage{tcolorbox} % Μορφοποίηση πλαισίων
\usepackage{titlesec} % Μορφοποίηση τίτλων κεφαλαίων
\usepackage{setspace} % Διάστιχο
\usepackage[greek]{babel} % Ελληνική γλώσσα
\usepackage{indentfirst} % Εσοχή πρώτης παραγράφου
\usepackage{fancyhdr} % Κεφαλίδες και υποσέλιδα
\usepackage{forest} % Δημιουργία δέντρων
\usepackage{graphicx} % Διαχείριση εικόνων
\usepackage{enumitem} % Διαμόρφωση λιστών
\usepackage{listings} % Block κώδικα
\usepackage{arabicore} %
\usepackage{booktabs} % πίνακες
\usepackage{titleps} % επικεφαλίδες και υποσέλιδα
\usepackage{biblatex} % διαχείριση βιβλιογραφίας
\usepackage{lstmisc} %
\usepackage{tikz}

\graphicspath{ {./img/} }
%\usepackage[fontsize=13pt]{fontsize}

\setmainfont{Conduit ITC Hel Light}
\newfontfamily\fontDin{CF Din Condensed}
    \newenvironment{Din}{\fontDin}{\par}
\newfontfamily\fontDinLight{CF Din Light Condensed}
    \newenvironment{Din Light}{\fontDinLight}{\par}
\newfontfamily\fontDinMedium{CF Din Medium Condensed}
    \newenvironment{Din Medium}{\fontDinMedium}{\par}
%\newfontfamily\fontCode{Courier New}
    %\newenvironment{Code}{\fontCode}{\par}
\setmonofont[Scale=0.8]{Courier New}
\newfontfamily\fontTimes{Times New Roman}
    \newenvironment{Times}{\fontTimes}{\par}

\newfontfamily\headingfont[]{CF Din Medium Condensed}

\addto\captionsgreek{% Replace "english" with the language you use
  \renewcommand{\contentsname}%
    {ΠΕΡΙΕΧΟΜΕΝΑ}%
}

\newtcolorbox{headerdark}{colback=darkgray, boxrule=0pt,arc=0pt, boxsep=12pt,left=2pt,right=2pt,leftrule=0pt}
\newtcolorbox{headerlight}{colback=gray!50, boxrule=0pt,arc=0pt, boxsep=2pt,left=2pt,right=2pt,leftrule=0pt}
\newtcolorbox{graycomment}{colback=gray!25, boxrule=0pt,arc=0pt, boxsep=2pt,left=2pt,right=2pt,leftrule=1pt, grow to left by=-28pt,  grow to right by=-28pt}

% Το στιλ των νέων κεφαλαίων/sectionσ
\titleformat{\chapter}[block]
    {\color{black} \fontDin \large } % global formatting (number and title)
    {\color{white} \fontDin \colorbox{black!90} {\hspace{7pt}\thechapter\hspace{7pt}}} % label: number and its formatting
    {} % spacing between number and title
    {\colorbox{gray!25}} % optional (content between number and title)
\titlespacing*{\chapter}
   {0pt}{1em}{0.4em}  % left before after
\titleclass{\chapter}{straight}

\titleformat{\section}[block]
    {\color{black} \fontDinLight \large} % global formatting (number and title)
    {\color{white} \fontDin \colorbox{black!90} {\hspace{5pt}\thesection\hspace{5pt}}} % label: number and its formatting
    {} % spacing between number and title
    {\colorbox{gray!25}} % optional (content between number and title)
\titlespacing*{\section}
   {0pt}{1em}{0.4em}  % left before after

\titleformat{\subsection}[block]
    {\fontDinLight \large} % global formatting (number and title)
    {\fontDin\colorbox{gray!25} {\hspace{5pt}\thesubsection\hspace{5pt}}} % label: number and its formatting
    {\hspace{10pt}} % spacing between number and title
    {} % optional (content between number and title)
\titlespacing*{\subsection}
   {0pt}{1em}{0.4em}  % left before after

%% Setting for the column separator
%\colorlet{shadecol}{black!20}
%\setlength\columnsep{12pt}
%\makeatletter   % This change vertical bar to a dotted line
%\newcommand{\latexcolumnseprulecolor}{\color{shadecol}}
%\renewcommand\dotfill[1][0.4em]{%
%  \leavevmode\cleaders\hb@xt@ #1{\hss .\hss}\hfill\kern\z@}
%\patchcmd{\@outputdblcol}%
%  {\vrule\@width\columnseprule}%
%  {\rotatebox{90}{\parbox{\textheight}{\dotfill[0.3em]}}}%
%  {}{}
%\makeatother
%
%% Command to render shaded heading
%\newsavebox\labbox
%\NewDocumentCommand\shadedsec{O{1.5\baselineskip} O{0.33\dimexpr#1} m}{%
%  \sbox\labbox{%
%    \colorbox{black!80}{%
%      \textcolor{white}{\hspace{0.5em}#3\hspace{0.5em}}}}
%  \addvspace{#1}
%  \noindent%
%  \nopagebreak%
%  \usebox\labbox
%  {\color{shadecol}\rule[-\fboxsep]{\dimexpr\columnwidth-\wd\labbox}{\dimexpr\ht\labbox+\fboxsep}}%
%  \vspace{#2}\par}


% Ρύθμση αλλαγής γραμμής
\tolerance=1
\emergencystretch=\maxdimen
\hyphenpenalty=10000 % Για να μην κάνει συλλαβισμό στις λέξεις
\hbadness=10000

% Αλλαγή απόστασης μεταξύ παραγράφων
\setlength{\parskip}{6pt}

% Ρύθμιση Headers/Footers
\pagestyle{fancy}
\renewcommand{\headrulewidth}{0pt}
\fancyhead{}\fancyfoot{}
\fancyhead[R]{\fontDinLight ΑΛΕΞΑΝΔΡΟΣ ΞΙΑΡΧΟΣ \(\cdot\) 1059619\hspace{10pt}\colorbox{darkgray}{\color{white}\fontDin\thepage}}

% Συνεχόμενη αρίθμηση ανά chapters
\counterwithout{footnote}{chapter}

% Για γραμμές κώδικα
\lstdefinestyle{mystyle}{
    backgroundcolor=\color{gray!10},
    keywordstyle=\bf\ttfamily,
    numberstyle=\tiny\color{darkgray},
    basicstyle=\ttfamily\footnotesize,
    breakatwhitespace=false,
    breaklines=true,
    captionpos=b,
    showstringspaces=false,
    keepspaces=true,
    numbers=left,
    numbersep=5pt,
}
\lstset{style=mystyle}

\usepackage{float} % Allows for more flexible positioning of floats σε πίνακες πχ
\restylefloat{table}

\endinput
