\chapter{ΕΡΩΤΗΜΑ 1}
    Για την εισαγωγή και τη προεπεξεργασία του \texttt{.csv} αρχείου, θα χρησιμοποιήσουμε τη βιβλιοθήκη \texttt{pandas} της Python.

    \section{ΑΝΑΛΥΣΗ ΣΥΝΟΛΟΥ ΔΕΔΟΜΕΝΩΝ}
        Καταρχάς, χρησιμοποιώντας την συνάρτηση \texttt{head()} μπορούμε να δούμε τις πρώτες 5 εγγραφές από το σύνολο δεδομένων μας:

        \begin{table}[ht] \noindent\centering\tt
        \resizebox{\textwidth}{!}{
        \begin{tabular}{llrrrrrrr}
        & timestamp & back\_x & back\_y & back\_z & thigh\_x & thigh\_y & thigh\_z & label \\
        \midrule
        0 & 2019-01-12 00:00:00.000 & -0.760242 & 0.299570 & 0.468570 & -5.092732 & -0.298644 & 0.709439 & 6 \\
        1 & 2019-01-12 00:00:00.010 & -0.530138 & 0.281880 & 0.319987 & 0.900547 & 0.286944 & 0.340309 & 6 \\
        2 & 2019-01-12 00:00:00.020 & -1.170922 & 0.186353 & -0.167010 & -0.035442 & -0.078423 & -0.515212 & 6 \\
        3 & 2019-01-12 00:00:00.030 & -0.648772 & 0.016579 & -0.054284 & -1.554248 & -0.950978 & -0.221140 & 6 \\
        4 & 2019-01-12 00:00:00.040 & -0.355071 & -0.051831 & -0.113419 & -0.547471 & 0.140903 & -0.653782 & 6 \\
        \end{tabular}}
        \end{table}

        Σε συνδυασμό με την \texttt{info()}, παρατηρούμε πώς για κάθε χρονική στιγμή δίνονται οι τιμές των αισθητήρων, αποθηκευμένες ως \texttt{float24},
        στις τρεις διαστάσεις (x, y, z) για τις περιοχές της πλάτης και του μηρού, καθώς και ένα \texttt{int64} label. Μέσω των \texttt{isna().sum()} συναρτήσεων παρατηρούμε πώς δεν υπάρχουν \texttt{NaN} τιμές στο σύνολο δεδομένων.

        Χρησιμοποιώντας τη συνάρτηση \texttt{describe()} μπορούμε να υπολογίσουμε βασικές στατιστικές μετρικές για τα δεδομένα μας:



    \section{ΓΡΑΦΙΚΕΣ ΠΑΡΑΣΤΑΣΕΙΣ}