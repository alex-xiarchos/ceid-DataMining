\chapter{ΕΡΩΤΗΜΑ 1}
    Για την εισαγωγή και τη προεπεξεργασία του \texttt{.csv} αρχείου, θα χρησιμοποιήσουμε τη βιβλιοθήκη \texttt{pandas} της Python.

    \section{ΑΝΑΛΥΣΗ ΣΥΝΟΛΟΥ ΔΕΔΟΜΕΝΩΝ}
        Καταρχάς, χρησιμοποιώντας την συνάρτηση \texttt{head()} μπορούμε να δούμε τις πρώτες εγγραφές από το πρώτο αρχείο του συνόλου δεδομένων \texttt{S006.csv}:

        \begin{table}[ht] \noindent\centering\tt
        \resizebox{\textwidth}{!}{
            \begin{tabular}{llrrrrrrr}
            & timestamp & back\_x & back\_y & back\_z & thigh\_x & thigh\_y & thigh\_z & label \\
            \midrule
            0 & 2019-01-12 00:00:00.000 & -0.760242 & 0.299570 & 0.468570 & -5.092732 & -0.298644 & 0.709439 & 6 \\
            1 & 2019-01-12 00:00:00.010 & -0.530138 & 0.281880 & 0.319987 & 0.900547 & 0.286944 & 0.340309 & 6 \\
            2 & 2019-01-12 00:00:00.020 & -1.170922 & 0.186353 & -0.167010 & -0.035442 & -0.078423 & -0.515212 & 6 \\
            3 & 2019-01-12 00:00:00.030 & -0.648772 & 0.016579 & -0.054284 & -1.554248 & -0.950978 & -0.221140 & 6 \\
            4 & 2019-01-12 00:00:00.040 & -0.355071 & -0.051831 & -0.113419 & -0.547471 & 0.140903 & -0.653782 & 6 \\
            \end{tabular}}
        \end{table}

        Σε συνδυασμό με την \texttt{info()}, παρατηρούμε πώς για κάθε χρονική στιγμή δίνονται οι τιμές των αισθητήρων, αποθηκευμένες ως \texttt{float24},
        στις τρεις διαστάσεις (x, y, z) για τις περιοχές της πλάτης και του μηρού, καθώς και ένα \texttt{int64} label.

        Για να ελέγξουμε την ακεραιότητα και να εντοπίσουμε τυχούσες συνέπειες, μέσω της συνάρτησης \texttt{concat()} ενώνουμε όλα τα 22 αρχεία σε ένα ενιαίο dataframe.

        \begin{table}[ht] \noindent\centering\tt
        \resizebox{0.2\textwidth}{!}{
            \begin{tabular}{lr}
             & sum \\
            \midrule
            timestamp & 0 \\
            back_x & 0 \\
            back_y & 0 \\
            back_z & 0 \\
            thigh_x & 0 \\
            thigh_y & 0 \\
            thigh_z & 0 \\
            label & 0 \\
            index & 5740689 \\
            Unnamed: 0 & 6323682 \\
            \end{tabular}}
        \end{table}

        Παρατηρούμε πως έχουν εμφανιστεί \texttt{NaN} τιμές στις στήλες \texttt{index} και \texttt{Unnamed: 0}, οι οποίες μάλιστα δεν ήταν παρούσες στο αρχικό \texttt{S006.csv}.
        Ελέγχοντας όλα τα αρχεία, η στήλη \texttt{index} εμφανίζεται στο αρχείο \texttt{} και η στήλη \texttt{Unnamed: 0} στο αρχείο \texttt{}.

        % ΜΕΤΑ ΤΟΝ ΚΑΘΑΡΙΣΜΟ ΤΩΝ ΑΡΧΕΙΩΝ
        Χρησιμοποιώντας τη συνάρτηση \texttt{describe()} μπορούμε να υπολογίσουμε βασικές στατιστικές μετρικές για τα δεδομένα μας.
        Παίρνοντας ως παράδειγμα το \texttt{S006.csv}:

        \begin{table}[ht] \noindent\centering\tt
        \resizebox{\textwidth}{!}{
            \begin{tabular}{lrrrrrrr}
             & back_x & back_y & back_z & thigh_x & thigh_y & thigh_z & label \\
            \midrule
            count & 408709 & 408709 & 408709 & 408709 & 408709 & 408709 & 408709 \\
            mean & -0.802201 & -0.000687 & -0.274718 & -0.370317 & 0.143471 & 0.617527 & 10.190187 \\
            std & 0.238347 & 0.189062 & 0.441805 & 0.506666 & 0.213864 & 0.536430 & 20.328336 \\
            min & -3.542889 & -3.016498 & -1.024363 & -6.844045 & -5.757406 & -4.884791 & 1.000000 \\
            25\% & -0.983647 & 0.001063 & -0.702338 & -0.952840 & 0.022534 & 0.144114 & 6.000000 \\
            50\% & -0.937195 & 0.033240 & -0.277446 & -0.277711 & 0.086248 & 0.924066 & 7.000000 \\
            75\% & -0.654541 & 0.074822 & 0.064811 & 0.068999 & 0.246292 & 1.001372 & 7.000000 \\
            max & 0.952109 & 2.569339 & 1.628023 & 3.898547 & 4.602909 & 5.391660 & 130.000000 \\
            \end{tabular}}
        \end{table}



    \section{ΓΡΑΦΙΚΕΣ ΠΑΡΑΣΤΑΣΕΙΣ}